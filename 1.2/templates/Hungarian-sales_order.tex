\documentclass[twoside]{article}
\usepackage[frame]{xy}
\usepackage{tabularx}
\usepackage[latin2]{inputenc}
\setlength{\voffset}{0.5cm}
\setlength{\hoffset}{-2.0cm}
\setlength{\topmargin}{0cm}
\setlength{\headheight}{0.5cm}
\setlength{\headsep}{1cm}
\setlength{\topskip}{0pt}
\setlength{\oddsidemargin}{1.0cm}
\setlength{\evensidemargin}{1.0cm}
\setlength{\textwidth}{19.2cm}
\setlength{\textheight}{24.5cm}
\setlength{\footskip}{1cm}
\setlength{\parindent}{0pt}
\renewcommand{\baselinestretch}{1}
\begin{document}

\newlength{\descrwidth}\setlength{\descrwidth}{10cm}

\newsavebox{\hdr}
\sbox{\hdr}{
  \fontfamily{cmss}\fontsize{10pt}{12pt}\selectfont

  \parbox{\textwidth}{
    \parbox[b]{12cm}{
      <?lsmb company ?>
      
      <?lsmb address ?>}\hfill
    \begin{tabular}[b]{rr@{}}
    Telefon & <?lsmb tel ?>\\
    Fax & <?lsmb fax ?>
    \end{tabular}

    \rule[1.5ex]{\textwidth}{0.5pt}
  }
}
    
\fontfamily{cmss}\fontshape{n}\selectfont

\markboth{<?lsmb company ?>\hfill <?lsmb ordnumber ?>}{\usebox{\hdr}}

\pagestyle{myheadings}
%\thispagestyle{empty}     use this with letterhead paper

<?lsmb pagebreak 65 27 48 ?>
\end{tabular*}

  \rule{\textwidth}{2pt}
  
  \hfill
  \begin{tabularx}{7cm}{Xr@{}}
  \textbf{R�sz�sszeg} & \textbf{<?lsmb sumcarriedforward ?>} \\
  \end{tabularx}

\newpage

\markright{<?lsmb company ?>\hfill <?lsmb ordnumber ?>}

\vspace*{-12pt}

\begin{tabular*}{\textwidth}{@{}lp{\descrwidth}@{\extracolsep\fill}rlrrr@{}}
  \textbf{Sz�m} & \textbf{Sz�veges le�r�s} & \textbf{Menny.} &
    \textbf{egys.} & \textbf{�r} & \textbf{Engedm�ny} & \textbf{�sszeg} \\
  & folytat�s az el�z� oldalr�l: <?lsmb lastpage ?> & & & & & <?lsmb sumcarriedforward ?> \\
<?lsmb end pagebreak ?>


\fontfamily{cmss}\fontsize{10pt}{12pt}\selectfont

\vspace*{2cm}

<?lsmb name ?>

<?lsmb address1 ?>

<?lsmb if address2 ?>
<?lsmb address2 ?>
<?lsmb end address2 ?>

<?lsmb city ?> <?lsmb state ?> <?lsmb zipcode ?>

<?lsmb if country ?>
<?lsmb country ?>
<?lsmb end country ?>

\vspace{3.5cm}

\textbf{V E V �} \parbox{0.3cm}{\hfill} \textbf{R E N D E L � S}
\hfill
\begin{tabular}[t]{l@{\hspace{0.3cm}}l}
  \textbf{Rendel�s d�tuma} & <?lsmb orddate ?> \\
<?lsmb if reqdate ?>
  \textbf{Lesz�ll�t�s} & <?lsmb reqdate ?> \\
<?lsmb end reqdate ?>
  \textbf{Sz�m} & <?lsmb ordnumber ?>
\end{tabular}

\vspace{1cm}

\begin{tabular*}{\textwidth}{@{}lp{\descrwidth}@{\extracolsep\fill}rlrrr@{}}
  \textbf{Sz�m} & \textbf{Sz�veges le�r�s} & \textbf{Menny.} &
    \textbf{egys.} & \textbf{�r} & \textbf{Engedm�ny} & \textbf{�sszeg} \\
<?lsmb foreach number ?>
  <?lsmb number ?> & <?lsmb description ?> & <?lsmb qty ?> &
    <?lsmb unit ?> & <?lsmb sellprice ?> & <?lsmb discount ?> & <?lsmb linetotal ?> \\
<?lsmb end number ?>
\end{tabular*}


\parbox{\textwidth}{
\rule{\textwidth}{2pt}

\vspace{0.2cm}

\hfill
\begin{tabularx}{9cm}{Xr@{}}
  \textbf{Nett� �sszeg} & \textbf{<?lsmb subtotal ?>} \\
<?lsmb foreach tax ?>
  <?lsmb taxdescription ?> alap <?lsmb taxbase ?>  <?lsmb tax ?>\\
<?lsmb end tax ?>
  \hline
  \textbf{Brutt� �sszeg} & \textbf{<?lsmb ordtotal ?>}\\
\end{tabularx}

\vspace{0.3cm}

\hfill
%  All prices in \textbf{<?lsmb currency ?>} funds.
Az �rak Forintban �rtend�k
\vspace{12pt}

<?lsmb if notes ?>
  <?lsmb notes ?>
<?lsmb end if ?>

}


\renewcommand{\thefootnote}{\fnsymbol{footnote}}

\footnotetext[1]{\tiny
}

\end{document}

