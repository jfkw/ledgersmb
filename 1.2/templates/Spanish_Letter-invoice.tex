% ve-invoice.tex
\documentclass[letterpaper,oneside]{article}
\usepackage[frame]{xy}
\usepackage{tabularx}
\usepackage[latin1]{inputenc}
%\usepackage{marvosym}  % Euro \EUR
\usepackage{fancyhdr}
\setlength{\topmargin}{0cm}
\setlength{\topskip}{0cm}
\setlength{\headheight}{0cm}
\setlength{\headsep}{0.5cm}
\setlength{\textheight}{24.2cm}
\setlength{\textwidth}{19cm}
\setlength{\oddsidemargin}{-1.4cm}
\setlength{\evensidemargin}{-1.4cm}
\setlength{\footskip}{1cm}

\setlength{\parindent}{0pt}
\renewcommand{\baselinestretch}{1}
\begin{document}

\newlength{\descrwidth}\setlength{\descrwidth}{13.0cm}

\newsavebox{\hdr}

%Or whatever font you want!
\fontfamily{cmss}\fontshape{n}\selectfont

\sbox{\hdr}{

\begin{minipage}[t]{0.6\linewidth}
\vspace{2.2cm}

\begin{tabular}[t]{p{1.7cm}p{2.4cm}p{1.7cm}}\\
\centering{N�mero} & \centering{Fecha} & C. Cliente\\
\centering{<?lsmb invnumber ?>} & \centering{<?lsmb invdate ?>} & \centering{<?lsmb customer_id ?>}
\end{tabular}
\end{minipage}

\begin{minipage}[t]{0.4\linewidth}
\textbf{F A C T U R A} 
\vspace{1cm}

<?lsmb name ?>

<?lsmb address1 ?>

<?lsmb address2 ?>

<?lsmb city ?> <?lsmb state ?> <?lsmb zipcode ?>

<?lsmb country ?>
\end{minipage}

}

\pagestyle{fancy}
\renewcommand{\headrulewidth}{0cm}
\renewcommand{\footrulewidth}{0cm}
\cfoot{\thepage}
%\markboth{\usebox{\hdr}}{\usebox{\hdr}}
%\thispagestyle{empty}     %use this with letterhead paper

<?lsmb pagebreak 65 27 37 ?>
\end{tabular*}
\newpage
\usebox{\hdr}
%\markboth{\usebox{\hdr}}{\usebox{\hdr}}
\vspace{0.5cm}

\begin{tabular*}{\textwidth}{rp{\descrwidth}rr}
  \textbf{Cant.} & \textbf{Descripci�n} & \textbf{Precio} & \textbf{Importe} \\ \hline
<?lsmb end pagebreak ?>

\fontfamily{cmss}\fontsize{10pt}{12pt}\selectfont

\usebox{\hdr}
\vspace{0.5cm}

\begin{tabular*}{\textwidth}{rp{\descrwidth}rr}
  \textbf{Cant.} & \textbf{Descripci�n} & \textbf{Precio} & \textbf{Importe} \\ \hline
<?lsmb foreach number ?>
  <?lsmb qty ?> & <?lsmb description ?> & <?lsmb sellprice ?> & <?lsmb linetotal ?> \\
<?lsmb end number ?>
\end{tabular*}

\parbox{\textwidth}{
\vspace{12pt}
<?lsmb if notes ?>
  <?lsmb notes ?>
<?lsmb end if ?>
}

\vfill

\begin{flushright}
\begin{tabularx}{10cm}{Xr@{}}
  \textbf{Base imponible} & \textbf{<?lsmb subtotal ?>} \\
<?lsmb foreach tax ?>
  IVA (<?lsmb taxrate ?>\%) sobre <?lsmb taxbase ?> & <?lsmb tax ?> \\
<?lsmb end tax ?>
  \hline
  \textbf{Total} & \textbf{<?lsmb invtotal ?>} \\
\end{tabularx}
\end{flushright}

%\renewcommand{\thefootnote}{\fnsymbol{footnote}}

\end{document}

