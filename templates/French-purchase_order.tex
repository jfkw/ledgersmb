\documentclass[twoside]{scrartcl}
\usepackage[frame]{xy}
\usepackage[frenchb]{babel}
\usepackage[latin1]{inputenc}
\usepackage{tabularx}
\setlength{\voffset}{0.5cm}
\setlength{\hoffset}{-2.0cm}
\setlength{\topmargin}{0cm}
\setlength{\headheight}{0.5cm}
\setlength{\headsep}{1cm}
\setlength{\topskip}{0pt}
\setlength{\oddsidemargin}{1.0cm}
\setlength{\evensidemargin}{1.0cm}
\setlength{\textwidth}{19.2cm}
\setlength{\textheight}{24.5cm}
\setlength{\footskip}{1cm}
\setlength{\parindent}{0pt}
\renewcommand{\baselinestretch}{1}
\begin{document}

\newlength{\descrwidth}\setlength{\descrwidth}{10cm}

\newsavebox{\hdr}
\sbox{\hdr}{
  \fontfamily{cmss}\fontsize{10pt}{12pt}\selectfont

  \parbox{\textwidth}{
    \parbox[b]{12cm}{
      <?lsmb company ?>
      
      <?lsmb address ?>}\hfill
    \begin{tabular}[b]{rr@{}}
    T�l�phone & <?lsmb tel ?>\\
    T�l�copieur & <?lsmb fax ?>
    \end{tabular}

    \rule[1.5ex]{\textwidth}{0.5pt}
  }
}
    
\fontfamily{cmss}\fontshape{n}\selectfont

\markboth{<?lsmb company ?>\hfill <?lsmb ordnumber ?>}{\usebox{\hdr}}

\pagestyle{myheadings}
%\thispagestyle{empty}     use this with letterhead paper

<?lsmb pagebreak 65 27 37 ?>
\end{tabular*}

  \rule{\textwidth}{2pt}
  
  \hfill
  \begin{tabularx}{7cm}{Xr@{}}
  \textbf{Sous-total} & \textbf{<?lsmb sumcarriedforward ?>} \\
  \end{tabularx}

\newpage

\markright{<?lsmb company ?>\hfill <?lsmb ordnumber ?>}

\vspace*{-12pt}

\begin{tabular*}{\textwidth}{@{}lp{\descrwidth}@{\extracolsep\fill}rlrr@{}}
  \textbf{Number} & \textbf{Description} & \textbf{Qt'y} &
    \textbf{Unit} & \textbf{Price} & \textbf{Amount} \\
  & report� de la page <?lsmb lastpage ?> & & & & <?lsmb sumcarriedforward ?> \\
<?lsmb end pagebreak ?>


\fontfamily{cmss}\fontsize{10pt}{12pt}\selectfont

\vspace*{2cm}

<?lsmb name ?>

<?lsmb address1 ?>

<?lsmb if address2 ?>
<?lsmb address2 ?>
<?lsmb end address2 ?>

<?lsmb city ?> <?lsmb state ?> <?lsmb zipcode ?>

<?lsmb if country ?>
<?lsmb country ?>
<?lsmb end country ?>

\vspace{3.5cm}

\textbf{B O N} \parbox{0.3cm}{\hfill} \textbf{D E} \parbox{0.3cm}{\hfill}
\textbf{C O M M A N D E}
\hfill
\begin{tabular}[t]{l@{\hspace{0.3cm}}l}
  \textbf{Date de la commande} & <?lsmb orddate ?> \\
<?lsmb if reqdate ?>
  \textbf{Livrable le} & <?lsmb reqdate ?> \\
<?lsmb end reqdate ?>
  \textbf{Num�ro de commande} & <?lsmb ordnumber ?>
\end{tabular}

\vspace{1cm}

\begin{tabular*}{\textwidth}{@{}lp{\descrwidth}@{\extracolsep\fill}rlrr@{}}
  \textbf{Num�ro} & \textbf{Description} & \textbf{Qt�} &
    \textbf{Unit�} & \textbf{Prix} & \textbf{Montant} \\
<?lsmb foreach number ?>
  <?lsmb number ?> & <?lsmb description ?> & <?lsmb qty ?> &
    <?lsmb unit ?> & <?lsmb sellprice ?> & <?lsmb linetotal ?> \\
<?lsmb end number ?>
\end{tabular*}


\parbox{\textwidth}{
\rule{\textwidth}{2pt}

\vspace{0.2cm}

\hfill
\begin{tabularx}{7cm}{Xr@{}}
  \textbf{Sous-total} & \textbf{<?lsmb subtotal ?>} \\
<?lsmb foreach tax ?>
  <?lsmb taxdescription ?> de <?lsmb taxbase ?> & <?lsmb tax ?>\\
<?lsmb end tax ?>
  \hline
  \textbf{Total} & \textbf{<?lsmb ordtotal ?>}\\
\end{tabularx}

\vspace{0.3cm}

\hfill
  Tous les prix indiqu�s sont en \textbf{<?lsmb currency ?>}.

\vspace{12pt}

<?lsmb if notes ?>
  <?lsmb notes ?>
<?lsmb end if ?>

}


%\renewcommand{\thefootnote}{\fnsymbol{footnote}}

%\footnotetext[1]{\tiny }

\end{document}
